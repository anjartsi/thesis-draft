\chapter{Literature Review}
\label{literature-review}
There are many AV systems that have been created, such s 
ANIMAL \cite{h}, HalVis \cite{i}, or BRIDGES \cite{g}. 
And yet, these systems fail to reach mainstream computer science education,
according to Hundhauser and Douglas \cite{j}, 
for a number of reasons, the greatest of which being that 
instructors find them too difficult to use , 
to the point where these systems take up more of the instructor's time 
in preparation than they save during class. 
And according to RoBling and Naps \cite{f}, there are eight 
pedagogical requirements for a successful AV System, which are summed up as such: 
\begin{itemize}
	\item General-purpose system that reaches a large target audience
	\item Allows the user to provide input, but in a manner that is not overly burdensome
	\item Allows the user to go backwards and forwards to different points of the 
		algorithm
	\item Allows (and encourages) the user to interact with the system, and make
		predictions about what the algorithm will do next. 
	\item Provides the user with textual explanations about what's going on
	\item Displays changing animations so the user can detect what happened. 
\end{itemize}
The observations made by \cite{j} and \cite{f}
served as the inspiration for my project, 
and I set out to create a tool that would be easy to use
in order to encourage instructors to incorporate it into their curricula. 

%\item {https://visualgo.net/en}
%\item {https://www.cs.usfca.edu/~galles/visualization/Algorithms.html}