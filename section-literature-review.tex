\chapter{Literature Review}
\label{literature-review}
There are many AV systems that have been created, such as 
ANIMAL \cite{h}, HalVis \cite{i}, or BRIDGES \cite{g} that 
have been shown to be more effective than traditional teaching methods. 
And yet these systems fail to reach mainstream computer science education. 
\newline\newline
Many of these AV systems look very old compared with modern applications, 
and they require to be downloaded and set up on a local machine before 
they can be used.
According to Hundhauser and Douglas \cite{j}, 
there are a number of reasons for the failure of 
AV's to have widespread educational use.
One of those reasons is that instructors find them too difficult to use, 
or too time-consuming to learn, 
taking up more of the instructor's time 
in preparation than they save during class. 
\newline\newline
These observations served as the inspiration for my project, 
where the main goal is to create a tool that is easy enough to use
that it encourages instructors to incorporate it into their curricula
because it will save them time (both in the classroom and out of it) 
and make their lectures more engaging to the students. 
\newline\newline
Another concern some instructors have shown, according to Hundhauser and Douglas,
is over the effectiveness of 
an AV system compared with more traditional teaching methods 
(such as writing on the whiteboard). 
Teachers are hesitant to invest their time learning a new system 
if they are not certain it will actually help their students learn.
%\item {https://visualgo.net/en}
%\item {https://www.cs.usfca.edu/~galles/visualization/Algorithms.html}
But according to RoBling and Naps \cite{f}, there are eight 
pedagogical requirements that make an AV system an effective learning tool.
I tried to design my own project in a way that 
would meet most of those requirements. 
The requirements are summed up as such: 
\begin{itemize}
	\item General-purpose system that reaches a large target audience
	\item Allows the user to provide input, but in a manner that is not overly burdensome
	\item Allows the user to go backwards and forwards to different points of the 
	algorithm
	\item Allows (and encourages) the user to interact with the system, and make
	predictions about what the algorithm will do next. 
	\item Provides the user with textual explanations about what's going on
	\item Displays changing animations so the user can detect what happened. 
\end{itemize}