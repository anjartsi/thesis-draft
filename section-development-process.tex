\chapter{Development Process}
%%%%%%%%%%%%%%%%%%%%%%%%%%%%%%%%%%%%%%%%%%%%%%%%%%%%%%%%%%%%%%%%%%%%%%%%%%%%%%%%%%%
Since I was working by myself on this project, I did not adopt 
a formal development process, but it was iterative, with multiple deliveries, 
and continued improvement. The nature of this project is very modular.
It consists of various algorithms which do not rely on one another.
Thus I worked on each algorithm individually 
from conception to completion and published it to the live website 
before moving on to the next algorithm. 
The process of creating an AV for a given problem went (roughly) as follows:
\begin{itemize}
	\item Decide acceptable ranges and restrictions for inputs and problem size
	\begin{itemize}
		\item These restrictions ensure that the app will run smoothly and not
			slow down
		\item The restrictions also make sure the visualization is 
			easy to understand, and fits on the target screen size
		\item The ranges should be larger than if the problem were being 
			solved by hand
	\end{itemize}
	\item Create the data model
	\begin{itemize}
		\item What data structures are required for a given type of problem
		\item What properties and methods are required to run an algorithm
		\item Which parts of the data should be made available to the display
	\end{itemize}
	\item Create the \textsc{Instance Maker}
	\begin{itemize} 
		\item User interface (text fields, buttons, sliders, etc.) to 
			construct an instance manually
		\item Save/Load interface through which the user can upload/download
			a text file to create an instance
		\item Predefined example instances that can be loaded without 
			manually being inputted (this is a nice-to-have)
		\item Update the data model as the user makes changes to the problem instance
	\end{itemize}
	\item Create the \textsc{Display}
	\begin{itemize}
		\item Visual components that represent the data model
		\item Must respond to changes in the data model
	\end{itemize}
	\item Create the \textsc{Solver}
	\begin{itemize}
		\item User interface to run the given algorithm
		\item Update the data model as the algorithm is performed
	\end{itemize}
	\item Show Professor Noga the progress
	\begin{itemize}
		\item Discuss various decisions/restrictions/compromises made
			and whether they are acceptable
		\item Make any necessary changes
	\end{itemize}
\end{itemize}