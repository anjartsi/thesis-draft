\chapter{Development Process}
%%%%%%%%%%%%%%%%%%%%%%%%%%%%%%%%%%%%%%%%%%%%%%%%%%%%%%%%%%%%%%%%%%%%%%%%%%%%%%%%%%%
Since I was working by myself on this project, I did not adopt 
a formal development process, 
but my development process has been iterative, with continuous deliveries, 
and continued improvement. The nature of this project is very modular:
it consists of various algorithms which do not rely on one another.
Thus I worked on each algorithm individually 
from conception to completion, published it to the live website 
and then I moved on to the next algorithm. 
The process of creating an AV for a given problem went (roughly) as follows:
\section{Choosing which Algorithms to Cover}

The pool of algorithms I chose from for this project
is those taught in Comp 482 at CSUN. 
Since the goal of this project is to create a tool to be used by CSUN faculty,
it makes sense to create AV's for algorithms already taught at CSUN. 
\newline\newline
\newline\newline
I chose \textbf{Interval Scheduling} for a number of reasons. 
First and foremost, it is taught very early in the semester so 
it can students are inexperienced.
Secondly, it is a Greedy Algorithm, and it drives a conversation
about what makes an algorithm greedy:
students suggest different greedy solutions to the problem 
(take the shortest interval first,
take the interval with the least number of overlaps first, etc.)
while the professor proves or disproves each suggested solution. 
Oftentimes the professor has to use a counterexample to disprove 
students' attempted solutions, and this is where an AV would be useful.
\newline\newline
Another reason I chose this algorithm was because students often 
misunderstand the problem, especially the part where the intervals 
are all weighted equally (the length of an interval does not matter, 
it is the total number of intervals that need to be maximized).
Since the problem is introduced as 
"imagine you have a resource that many people want to use ..."
some students tend to think they must fill up as much time as possible 
in order to make efficient use of the resource
(while other students think they must fill up as little time as possible
to avoid wear-and-tear on the resource). 
This creates a situation where the teacher correcting the first group's
misunderstanding only reinforces the belief of the second group and vice versa.
For this reason, when creating the Interval Scheduling AV, this app
specifically provides a counter labeled "Intervals in Solution", which 
the teacher can point to and say "This is the number we want to maximize, nothing else!"
\newline\newline

\begin{itemize}
	\item Think about the problem in question
	\begin{itemize}
		\item What is the formal definition
		\item How is it introduced in lecture
		\item How is it performed as part of homework/test/lecture
		
	\end{itemize}
	\item Decide acceptable ranges and restrictions for inputs and problem size
	\begin{itemize}
		\item These restrictions ensure that the app will run smoothly and not
			slow down
		\item The restrictions also make sure the visualization is 
			easy to understand, and fits on the target screen size
		\item The ranges should be larger than if the problem were being 
			solved by hand
	\end{itemize}
	\item Create the data model
	\begin{itemize}
		\item What data structures are required for a given type of problem
		\item What properties and methods are required to run an algorithm
		\item Which parts of the data should be made available to the display
	\end{itemize}
	\item Create the \textsc{Instance Maker}
	\begin{itemize} 
		\item User interface (text fields, buttons, sliders, etc.) to 
			construct an instance manually
		\item Save/Load interface through which the user can upload/download
			a text file to create an instance
		\item Predefined example instances that can be loaded without 
			manually being inputted (this is a nice-to-have)
		\item Update the data model as the user makes changes to the problem instance
	\end{itemize}
	\item Create the \textsc{Display}
	\begin{itemize}
		\item Visual components that represent the data model
		\item Must respond to changes in the data model
	\end{itemize}
	\item Create the \textsc{Solver}
	\begin{itemize}
		\item User interface to run the given algorithm
		\item Update the data model as the algorithm is performed
	\end{itemize}
	\item Show Professor Noga the progress
	\begin{itemize}
		\item Discuss various decisions/restrictions/compromises made
			and whether they are acceptable
		\item Make any necessary changes
	\end{itemize}
\end{itemize}