\chapter{Requirements} 
\label{analysis-and-requirements}
%The requirements for this project were derived from a combination of 3 sources:
%\begin{itemize}
%	\item My personal experience as a high school teacher
%	\item Conversations with Professor Noga about what features he would like
%	\item The results of the studies discussed in 
%		\hyperref[literature-review]{\textbf{Section \ref{literature-review} Literature Review}}
%\end{itemize}
%%%%%%%%%%%%%%%%%%%%%%%%%%%%%%%%%%%%%%%%%%%%%%%%%%%%%%%%%%%%%%%%%%%%%%%%%%%%%%%%%%%
The requirements for this project were derived from a combination of three sources. 
First, I relied on my personal experience as a former teacher: 
I have taught Math and Science at the high school level, 
and would often incorporate technology into my lectures 
partly because I'm not good enough at drawing, 
and also because the students responded better to anything that was on a screen. 
\newline\newline
The second source for these requirements came from Professor Noga's input
since he is the faculty member who most frequently teaches Algorithms, 
I discussed with him how he approaches teaching each algorithm, what would be a 
useful feature of the app to be used during lecture, 
what maximum or minimum conditions need to be met by the app, and
what are some interesting instances that can stimulate in-class discussions. 
\newline\newline
Finally, I incorporated as much of what I learned from my research
(discussed in Chapter \ref{literature-review}, Literature Review)
to create an AV tool that would improve student learning. 
For example, AV's that allow the user to create their own instances 
and run te algorithm step by step are much more effective. 
\section{Requirements}
\label{requirements}
\begin{enumerate}
	\item Create a web-app for Algorithm Visualization 
	\begin{enumerate}
		\item The app will be a Single-Page Application
		\item The app will run in the front-end
		\item The app will run in the Google Chrome web browser
	\end{enumerate}
	\item The target users of The app will be CSUN faculty 
	who are teaching an intermediate-level 
	algorithms class (such as Comp 482)
	\item The app must be easy to use by its target users
	\begin{enumerate}
		\item Users should be able to interact with the app with 
			little to no formal training
		\item Where necessary, the app must provide instructions on 
			any controls that do not meet the above criteria
	\end{enumerate}
	\item The app will provide AV's for 3-5 algorithms 
		covered in such a class, including: 
		\begin{enumerate}
			\item Stable Matching (Stable Marriage)
			\item Interval Scheduling (unweighted)
			\item Closest Pair of Points
		\end{enumerate}
	\item For each algorithm, an AV will have the following:
	\begin{enumerate}
		\item \textsc{Instance Maker}: an interface to create/modify instances of the problem
		\begin{enumerate}
			\item User controls displayed on the webpage (text boxes, buttons, etc.)
			\item Loading an instance from a .txt file
			\item Saving an instance into a .txt file
			\item Loading an instance from a database
		\end{enumerate}
		\item \textsc{Solver}: an interface to solve the created instances (ie run the algorithm)
		\begin{enumerate}
			\item Perform a single step of the algorithm at a time
			\item Perform the entire algorithm automatically.
		\end{enumerate}
		\item \textsc{Display}: a visual diagram of an instance of the problem
		\begin{enumerate}
			\item The diagram must be organized and easy to understand
			\item The diagram must be appropriately sized to be displayed
				on a projector in front of students
			\item The display must update as the algorithm is performed
		\end{enumerate}
	\end{enumerate}
\end{enumerate}
%\subsection{Analysis and Architecture}
%(in progress)