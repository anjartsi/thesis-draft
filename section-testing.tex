\chapter{Testing and Validation}
\label{testing}
The two main goals of this project are to create an AV tool, and to create a tool 
that would be used by lecturers in an Algorithms classroom. 
To test for the first goal, I performed unit testing on the Data Models for 
each of the implemented algorithms.
To test for the second goal, I did live demonstrations of the app in front 
of live classrooms. 
\section{Unit Testing}
\hspace{-0.26in}
Unit tests were performed on the Data Models of each algorithm to ensure 
that the algorithms were being implemented correctly. 
The Vuex Store is where the Data Model is held, so each component of 
the Vuex Store was unit tested.
Each mutation and action in the store was tested with correct inputs as well as incorrect inputs
to make sure that the system would handle errors correctly. 
\newline\newline
Since all the functionality was implemented in the Data Model, and the 
Components simply display the data that is held in the Vuex Store, 
no automated tests were done for the user interface, but I tested each 
Vue Component manually to ensure it demonstrated the correct behavior. 
This manual testing was done on the Google Chrome browser, though I do not anticipate
any problems arising if Mozilla Firefox is used instead. 
\section{In-Class Presentations}
\subsection{Stable Marriage}
\hspace{-0.26in}
I demonstrated the Stable Marriage AV to two different sections of Professor Noga's 
Comp 482 class. 
The class had already been lectured about the Stable Marriage problem during the previous week, 
so they already had a solid understanding of the problem beforehand. 
I had a few problem instances that I had saved on my computer in advance, 
I used those instances to discuss some subtle intricacies of the Stable Marriage problem, 
such as:
\begin{itemize}
	\item In some instances the women's preference lists have no effect on the outcome
	\item Some instances can have multiple stable matchings
	\item Does the order in which the men propose affect the outcome?
	\item What is the lowest number of proposals that can happen? What instance can produce that result?
	\item What is the highest number of proposals that can happen? What instance can produce that result?
\end{itemize} 
Using the tool allowed me to discuss these ideas with the students, 
show them why some properties hold true while others don't. 
And on more than one occasion I repeated an example when I noticed
that some students were still confused.
\subsection{Interval Scheduling}
\hspace{-0.26in}
Similar to the Stable Marriage demonstration, I performed a lecture 
to two sections of the Comp 482 class the week after they had learned 
Interval Scheduling, so most of them had a decent understanding of the material. 
But more than a few of them were 
still confused about the problem (more than one student did not 
know that the problem asks for the total number of intervals).
After seeing the AV perform, these students seemed to understand the problem 
much better. 
\newline\newline
I used the app to perform a discussion-heavy lecture on 
various possible greedy solutions to the problem 
(take the shortest interval, or the interval with the least number of overlaps), 
and why many of those solutions would not produce the optimal result 
(``Because here's a counterexample!'').
I then used the AV to discuss with the class how an ``always ahead'' 
argument can be used to prove an algorithm's optimality. 
\newline\newline
My feeling after all these lectures was that students definitely appreciated 
the use of AV's in a classroom. Some of this could be attributed to 
the fact that it was a change of pace from their normal lectures, but 
I definitely think having the AV helped me as the lecturer to 
express my thoughts more clearly. 
