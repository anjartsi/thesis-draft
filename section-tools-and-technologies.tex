\chapter{Tools and Technologies}
\label{tools-and-technologies}
I am going to assume the reader has a basic understanding of HTML, CSS, and JavaScript.
When I first began working on this project, I made the decision to work in plain 
JavaScript, but I quickly realized that using a framework would allow me to focus 
on the more interesting parts of the project (how to visualize algorithms) without 
having to spend hours on the smaller problems 
(how to update dozens of elements on a page whenever a part of the data model changes).
So I chose to use Vue.js for my project. 
%%%%%%%%%%%%%%%%%%%%%%%%%%%%%%%%%%%%%%%%%%%%%%%%%%%%%%%%%%%%%%%%%%%%%%%%%%%%%%%%%%%
\section{Vue js}
\hspace{-0.25in}Vue was released in February 2014, by Evan You  \cite{vue-launch}. 
It is an open source JavaScript framework
for creating Single-Page Applications, but 
it can also be easily incorporated into an existing project. 
The two main features of Vue are creating reactive web pages, and components.
\newline\newline
For a web page to be \textbf{reactive}, it needs to respond to changes. This means
that various elements of the web page have to re-render themselves, whenever
parts of the data model change, and conversely the data model has to update whenever
a user interacts with the web page. 
This can be achieved in plain JavaScript by manually manipulating the elements
when the data model changes and using event listeners to handle user interactions, 
but it is very tedious, error-prone, and inefficient. 
Vue utilizes a virtual DOM (Document-Object Model) to efficiently 
render web pages and re-render only the parts of a page
that need to respond to changes in the data model, 
with minimal effort by the developer (me).
\newline\newline
The other key feature provided by Vue is \textbf{components}, which are 
self-contained parts of an application. 
Components are reusable, can be nested inside one another, and inherit from one another.
This makes Vue great for separating different tasks and responsibilities
of a program in an organized way. 
Vue uses a \textbf{template} syntax for its components, where each component 
has an HTML, CSS, and JavaScript portion. These can all be placed in the same file 
(but divided into \texttt{<template>}, \texttt{<style>}, and \texttt{<script>} sections respectively),
or each can go into a separate file. 
This template syntax was the biggest reason I chose Vue over alternative frameworks.
\newline\newline
Below is an example of a Vue component.
%
\lstinputlisting[
	caption = {Example Vue Component},
	label = {example.vue},
	style = {line-numbers}
	]{code-snippets/example.vue}
%
In the above example, the code between the  \texttt{<template>} tags 
is the HTML code, where the various elements are drawn on the web page.
The only new Vue-specific concepts in this section are on lines 5, 6, and 7. 
On line 5, the \texttt{\{\{~num~\}\}} inside the \texttt{<p>} tags will be replaced by 
value of the \texttt{num} variable associated with this Vue component (see next paragraph).
The \texttt{@click}s on lines 6 and 7 define what function is called 
when the user clicks on either button. 
The functions must be defined for this Vue component (again, see next paragraph). 
\newline\newline 
The code between the \texttt{<script>} tags is the JavaScript code,
where the data and functionality are defined. Here, the value for \texttt{num}
is initialized to zero (line 16), and the two functions called 
\texttt{increment} and \texttt{decrememnt} (lines 20 and 23)
increase or decrease the value of \texttt{num} by one. 
When either button gets pressed, these functions are called, 
the value of \texttt{num} changes, 
and the text inside the \texttt{<p>} tag is updated and re-rendered automatically.
\newline\newline
The code in the \texttt{<style>} tags is the CSS, where the 
elements' style, size, and positioning are set. 
The \texttt{scoped} keyword tells Vue that any style rules written here will not affect
other components. This greatly reduces the complexity of styling large applications
because the style rules for each component is located within the component itself. 
And when global style rules are required, they can be placed in the top-level 
component with the \texttt{scoped} keyword removed, and those rules will trickle 
down to all the nested components. 
%%%%%%%%%%%%%%%%%%%%%%%%%%%%%%%%%%%%%%%%%%%%%%%%%%%%%%%%%%%%%%%%%%%%%%%%%%%%%%%%%%
\section{Alternatives to Vue}
\hspace{-0.25in}The two most popular alternatives to Vue are React and Angular 
\cite{angular-react-vue}. 
Both of these frameworks are older than Vue, and they are used and maintained by 
tech giants Facebook and Google respectively. 
In this section I will briefly compare and contrast Vue with these frameworks and 
discuss why I chose to use Vue over either of them. 
\newline\newline
React and Vue are very similar: 
they both use reusable components to separate a program into self-contained units,
and they both utilize the Virtual DOM to create reactive webpages. 
But unlike Vue, where the HTML code is separated from the JavaScript,
React combines the two together using JSX
(an XML-like syntax for writing HTML code directly into JavaScript functions). 
The main reason I chose Vue over React was because I found using JSX to be 
much harder to follow than Vue's template syntax.
\newline\newline
Vue was inspired by Angular, so the two frameworks have a very similar syntax 
.or example: there is a command in Angular called \texttt{ng-if},
and its counterpart in Vue is called \texttt{v-if}.
But Angular is aimed at projects with much bigger scope than mine, 
and has a steeper learning curve \cite{angular-react-vue}.
I chose Vue over Angular because it is more lightweight than Angular \cite{vue-comparison}.
%%%%%%%%%%%%%%%%%%%%%%%%%%%%%%%%%%%%%%%%%%%%%%%%%%%%%%%%%%%%%%%%%%%%%%%%%%%%%%%%%%%
\section{Vuex}
\hspace{-0.26in}
Vue allows for top-level components to send data (or bind variables) down to any components that are 
nested within them. However, this data binding is only one-way, so the inner components
cannot send data back up to their parents. 
Vue provides a way to achieve this with events:
child components can emit events that their parents listen for, and can respond to.
Although this is functionally equivalent to having two-way data binding, it becomes much more verbose. 
Furthermore, the verbosity is compounded when sibling components 
(or worse yet, when elements that are much further apart in the tree)
need to communicate with one another. 
This is the reason I added \textbf{Vuex} \cite{vuex} to my project. 
\newline\newline
Vuex is a state-management library for Vue. It moves the responsibility of keeping track of 
data away from the components into a \textbf{store} whose sole purpose is to manage the data
in a structured way. A Vuex store is made up of a \textit{state}, \textit{mutations}, and \textit{actions} (there are other parts to Vuex stores like \textit{getters}, \textit{modules}, and \textit{plugins}, but I will not discuss them here).
\newline\newline
The \textit{state} is where all the data is kept for the entire application. 
This data can be accessed by any of the Vue components, 
but it can \underline{not} be modified by any of the Vue components. 
The only way to modify any data held in the state is to use \textit{mutations}. 
Mutations are functions that modify the state. 
\textit{Actions} are used as an interface between Vue and Vuex. 
Actions are functions that are ``dispatched'' from Vue components, 
and their main task is to invoke (``commit'') the mutations.
\newline\newline
I utilize Vuex for my project by creating a different store for each algorithm:
the \textit{state} holds all the information about the problem instance, 
the \textit{mutations} holds all the operations to edit the instance 
as well as all the operations that will be done to run the algorithm, 
and the \textit{actions} correspond to various tasks the user may wish to perform.
As an example, the Vuex store for Stable Marriage has the following: 
\begin{itemize}
	\item state
	\begin{itemize}
		\item Two $n \times n$ arrays where the preferences of the men and women are kept
		\item An $n \times n$ array of rejections (which women have rejected which men)
		\item A list ($0\leq length \leq n$) to keep track of what tentative matchings currently exist. 
	\end{itemize}
	\item mutations
	\begin{itemize}
		\item \texttt{swapPreferenceBoxes}: reorder the preference array
		\item \texttt{propose}: a man makes a proposal to a woman
		\item \texttt{acceptProposal}: the woman accepts the proposal
		\item \texttt{rejectProposal}: the woman rejects the proposal
		\item \texttt{resetSolver}: reset the algorithm to start from the beginning
	\end{itemize}
	\item actions
	\begin{itemize}
		\item \texttt{proposeDispose}: This is called when the user clicks a particular button in the solver
							to run a single step of the Gayle-Shapely algorithm.
		\item \texttt{loadFile}: This is called when the user loads a text file instead of manually 
							editing the problem instance. 
	\end{itemize}
\end{itemize}
Using Vuex to handle state management removes that responsibility from the Vue components,
making their sole responsibility to create a user interface. 
This separation of responsibilities not only makes the Vue components more reusable
it also makes the runtime process of each algorithm easier to 
implement, debug, and test.