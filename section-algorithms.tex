\chapter{The Algorithms} 
In this section I will discuss each of the algorithms included in the project 
with a formal definition, reasons why I chose each algorithm, and scope and limitations for each.
\section{Stable Marriage}
\subsection{Formal Definition}
Given $n$ men and $n$ women with a preference list, 
find a stable matching
\subsection{Scope and Limitation}
\hspace{-0.26in}
Instances of Stable Marriage where $n > 4$ tend to be too complex to solve by hand. 
Most in-class examples are problems with size $n = 3$.
This AV allows for problem sizes up to $n = 14$.
The reason behind this limit is to ensure the web page 
does not become overly large or overly slow. 
However, I recommend keeping the problem below $n = 6$ because that is 
the largest problem size that will still have the preference rows fit on a single line. 
\newline\newline
There are many variations of the Stable Marriage problem, for example 
having unequal numbers of men and women,
or having some people prefer to stay unmarried as a valid choice.
I only covered the basic version of the Stable Marriage problem, where 
there are an equal number of men and women, and every person must be married. 
\subsection{Why I chose to include this Algorithm}
\hspace{-0.26in}
I knew I wanted to cover Stable Marriage when I first decided on this project
because it is taught during the first week of class.
I gave precedence to subjects covered earlier in the semester 
because at that is the time when students neither have a firm grasp of the subject matter 
nor are they used to their professor's teaching style, 
so they are far more likely to have trouble understanding the material
than later on in the semester. 
\newline\newline
Another reason why I chose Stable Marriage was because of how tedious it is to 
solve this problem on paper, which (speaking from experience) involves many iterations of 
writing a matching and then crossing it out. 
Whereas having an AV tool to do the same task removes the busy work and allows 
the user to focus on the more important aspects of the problem.
%
\section{Interval Scheduling}
\subsection{Formal Definition}
\subsection{Scope and Limitation}
\subsection{Why I chose to include this Algorithm}
%
\section{Closest Pair of Points}
\subsection{Formal Definition}
\subsection{Scope and Limitation}
\subsection{Why I chose to include this Algorithm}