\chapter{Introduction} 
%%%%%%%%%%%%%%%%%%%%%%%%%%%%%%%%%%%%%%%%%%%%%%%%%%%%%%%%%%%%%%%%%%%%%%%%%%%%%%%%%%%
\section{Motivation}
\hspace{-0.26in}
When I was a high school Math and Science teacher,
I often tried to employ the use of technology in the classroom. 
I found it to be a good way to spend less class-time on tasks that do not 
engage the student. An example of such a task is drawing graphs in an Algebra class:
while I was busy drawing a diagram or graph, students would not be engaged in the lesson
and would start distracting their friends and classmates.
\newline\newline
So I started preparing graphs on a computer to display over a projector, and saved
valuable minutes of class time for more engaging activities. 
A side benefit of using computer-generated images is that they look far better
than any graph I can ever draw by hand. 
The only problem with this solution was that 
I had to spend more of my own time outside of class to create these visuals, or find
them online.
\newline\newline
Due to my background as a teacher, 
I was drawn towards a thesis project where I could create 
an educational tool to be used in a classroom. 
I wanted to help instructors draw diagrams and display them 
over a projector instead of having to draw them on the whiteboard.
The overall goal of this project is to create a tool that will save time in class 
by reducing the teacher's non-engaging tasks, while not requiring too much of the 
teachers' time outside of class for preparation. For the subject matter, I chose 
Algorithm Visualization.
%%%%%%%%%%%%%%%%%%%%%%%%%%%%%%%%%%%%%%%%%%%%%%%%%%%%%%%%%%%%%%%%%%%%%%%%%%%%%%%%%%%
\section{Background}
\label{introduction}
\hspace{-0.26in}
Algorithms are a fundamental part of Computer Science education. 
Just about every introductory CS class discusses the various sorting algorithms, 
their advantages and disadvantages, followed by some video that graphically 
shows how each sorting algorithm is implemented
%such as [\ref{sorting-out-sorting}]
\newline\newline
Algorithm Visualization (AV) is the use of software 
to create diagrams, animations, and other visual tools 
to facilitate the learning of an algorithm, 
its process (how it works and why it produces the correct result), 
and its complexity (runtime, required space, etc).
There is an abundance of visualizations for sorting algorithms,
and a decent amount for other introductory algorithms or data structures
(Binary Search Trees, Linked Lists, Kruskal's Algorithm, Primm's Algorithm, etc.),
but AV's are fairly few and far between for the intermediate-level algorithms. 
\newline\newline
Studies have shown the positive effects of various AV's 
(see {\textbf{Chapter \ref{literature-review} Literature Review}}),
and yet they are seldom used as part of the teaching process.
AV can provide advantages to instructors because it allows them to 
display complex diagrams or data structures without having to draw them, 
which not only saves class time for more engaging activities, but also
allows instructors to go over examples that are 
much too complex to draw by hand, such as a graph with more than 20 vertices, 
or a 2-dimensional grid containing 500 points. 
Furthermore, AV is helpful to students because it gives them a chance to 
review the lesson outside of class, reproducing the problem on their own
and giving them access to guided practice. 
\newline\newline
The Objectives for this project were influenced by the advantages discussed in the previous paragraph. 
These objectives outline the goals of the project as a whole, but 
a more detailed list of requirements can be found in
\textbf{Chapter \ref{requirements} Requirements}
%%%%%%%%%%%%%%%%%%%%%%%%%%%%%%%%%%%%%%%%%%%%%%%%%%%%%%%%%%%%%%%%%%%%%%%%%%%%%%%%%%%
\section{Objectives}
\begin{enumerate}
	\item Create an AV tool to be used by CSUN Faculty 
		teaching an intermediate Algorithms class during lecture
		\begin{enumerate}
			\item Algorithms covered by this project will be drawn from 
				those taught in Comp 482
			\item The tool will be used on a projector display as part of a lecture
		\end{enumerate}
	\item The tool must be intuitive and easy to use
		\begin{enumerate}
			\item Users should be able to use the tool with little or no 
				formal training
			\item Using the tool to create diagrams should take no more time 
				than drawing a similar diagram by hand
		\end{enumerate}
	\item The tool will have the following features:
	\begin{enumerate}
		\item Allow the user to create instances of a given problem
		\item Allow the user to simulate the steps of an algorithm 
			and see the solution
		\item Display visuals to describe how the algorithm is running
	\end{enumerate}
\end{enumerate}
%%%%%%%%%%%%%%%%%%%%%%%%%%%%%%%%%%%%%%%%%%%%%%%%%%%%%%%%%%%%%%%%%%%%%%%%%%%%%%%%%%%
\section{My Approach}
\hspace{-0.25in}The ultimate goal of this project is to create a tool that will actually be used 
in the classroom. The target users are CSUN faculty, and the target environment 
is in a classroom during lecture.
I kept this in mind when I was planning, designing, and testing my project. 
Whenever a decision had to be made between 
making the tool more universally applicable 
versus making the tool a better fit for the specific target audience,
I chose the latter. For example: that is why I chose 
specific lessons that were covered in CSUN's Comp 482 class.
\newline\newline
In order to make the the tool easily accessible (by faculty as well as students), 
it was designed as a web app, more precisely as a front-end single-page application:
\begin{itemize}
	\item Front-end: the app runs on the client's local machine
	\item Single-Page application: when the user interacts with the app,
		it changes the page dynamically without having to reload or refresh 
	\item Application: computer program that accomplishes some task
\end{itemize}
Each algorithm within the app is a self-contained module, so
it does not rely on other modules.
Each module is comprised of:
\begin{itemize}
	\item An \textsc{Instance Maker}: allows the user to create 
		instances of a given problem.
	\item A \textsc{Solver}: allows the user to run the algorithm.
	\item A \textsc{Display}: allows the user to see the various aspects
		of the problem while the algorithm is being performed.
\end{itemize} 
I used a JavaScript framework called Vue with an MVC approach.
I chose Vue because it is a relatively new framework and 
its popularity is on the rise \cite{top-js-libraries}.
Furthermore, I used a popular Vue library called Vuex, which is 
a state-management library \cite{vue}.
I discuss Vue and Vuex more in-depth in
\textbf{Chapter \ref{tools-and-technologies} Tools and Technologies}
I hosted the app on Heroku, because they offer free hosting for hobby-level apps. 
The app can currently be accessed by going to the following web site:
\begin{center}

\underline{https://funwithalgorithms.herokuapp.com}
	
\end{center}
%%%%%%%%%%%%%%%%%%%%%%%%%%%%%%%%%%%%%%%%%%%%%%%%%%%%%%%%%%%%%%%%%%%%%%%%%%%%%%%%%%%
%Revisit the Intro after writing the other sections
