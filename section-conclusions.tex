\chapter{Conclusions and Future Work}
Algorithm Visualization is an educationally effective method of teaching
how algorithms work. But educators tend to stick with the method they're used to: 
drawing diagrams on the whiteboard. 
Using AV's would save in-class time for discussion,  
produces higher-quality diagrams,
and is reproducible outside of the classroom.
\newline\newline
The goal of this project was to create an Algorithm Visualization tool to be used 
by lecturers to simplify the task of drawing complex diagrams. 
The tool is front-end web application that can be accessed by going to 
\underline{https://www.funwithalgorithms.com/}
\newline\newline
I chose three algorithms taught in CSUN's Comp 482 class:
Stable Marriage, 
Interval Scheduling, 
and Closest Pair of Points.
Each AV consists of three main components:
an \textsc{Instance Maker}, 
a \textsc{Display},
and a \textsc{Solver}. 
The \textsc{Instance Maker} allows users to 
create instances of a given problem by interacting with the page. 
The \textsc{Display} shows diagrams depicting the various 
properties of the instance. 
The \textsc{Solver} runs the algorithm step by step, 
modifying the \textsc{Display} in the process. 
\newline\newline
One of the biggest challenges of this project was 
separating responsibilities among the various components 
in an organized manner. 
Vue.js was used to keep isolate responsibilities into components, 
and Vuex was used to create an organized way for the components 
to communicate with each other. 
\newline\newline
There are plenty of opportunities to build upon this project, 
the most obvious of which is to add more AV's for different algorithms 
(the home page of the app lists quite a few good candidates). 
In an ideal scenario, the app would have an AV for each major 
algorithm covered in an entire course. 
\newline\newline
Aside from adding more AV, this project can be built-upon 
by adding more features to each AV itself for either functionality or 
to create a more fluid user interface. 
For example, the Stable Marriage AV could have a section where
the user (after creating the instance) checks to see whether a 
matching is stable. 
The Interval Scheduling could allow the user to reorder the 
intervals into different rows.
The Closest Pair of Points could allow the user to zoom in or 
out of the problem tree, and have a mini map to show them 
which part of the tree they are looking at. 
