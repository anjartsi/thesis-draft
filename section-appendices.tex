\chapter{Glossary}
\begin{itemize}
	\item \textbf {Algorithm} 
	~~$\to$~~
	A series of steps to solve a problem. 
	An algorithm must always terminate (it cannot run forever).
	
	\item \textbf {Algorithm Visualization (AV)}
	~~$\to$~~
	Software that displays diagrams, animations, and other interactive tools 
	to facilitate the learning of an algorithm. An AV is more interactive than an 
	animation or a video. 
	
	\item \textbf {App  / Page / Program / Tool / Web Page}
	~~$\to$~~
	I use these terms interchangeably to refer to the program 
	that I have created for this project. 
	
	\item \textbf {Front-End}
	~~$\to$~~
	A front-end web application runs on the client's machine instead 
	of a server. 
		
	\item \textbf {Instructor / Lecturer / Professor /User}
	~~$\to$~~
	I use these terms interchangeably to refer to the person that is using
	the app. This is not to say that student
	(or other person with no relation to the class)
	cannot use the app (it is available on the Internet for all to enjoy!)
	
	\item \textbf {Problem}
	~~$\to$~~
	The word problem is used in this document in the context 
	of a \textit{Mathematical Problem}.  
	Problems are different than algorithms, but students often use 
	the two words interchangeably. 
	\subitem As an example: Interval Scheduling is a \textbf{Problem}, 
	Earliest-Finishing-Time is an \textbf{Algorithm} that can be used to
	solve the Problem.
	
	\item \textbf {Problem Instance}
	~~$\to$~~
	A specific example of a problem with an input. 
	Homework questions and test questions often describe a problem instance
	and ask the student to find the solution to that instance.
\end{itemize}
%

%
